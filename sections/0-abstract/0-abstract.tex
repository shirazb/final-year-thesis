\small
Deep Learning has proven to be extremely powerful in variety of domains.
However, research is hindered by systems problems, such as the high memory requirement of training.
Checkpointing can help alleviate this by dropping forwards and recomputing them in the backwards pass; trading compute for memory.
The question then becomes of finding the optimal checkpointing \textit{policy} - what forwards to checkpoint and how many times to recompute.

\(\sqrt{N}\) checkpointing is the technique most commonly used.
It indiscriminately applies the policy that minimises the memory cost of an \(N\) layer network, rather than flexibly finding the policy that utilises all avaialble memory to minimise compute.
It also assumes uniform per-layer costs and is limited to one recomputation, so cannot handle very low memory budgets. 

A dynamic programming technique has been proposed for RNNs that achieves this.
As it is for RNNs, it too assumes uniform per-layer compute and memory costs.
Furthermore, to the best of my knowledge, the technique is not yet implemented in any of the popular deep learning frameworks.

Therefore, in this thesis, I first generalise the dynamic programming technique for RNNs to take into account the precise per-layer costs obtained from profiling.
The policy solver can then be much more judicious in determining how exactly to minimise compute whilst still satisfying the memory budget; resulting in lower compute costs for the same budget as before, or the ability to reach even lower budgets.
I show how this is quite non-trivial due to the now \textit{continuous} memory state.
Also, as a prerequisite to this, I meticulously derive the computational graph of network training and analyse its memory requirements, so as to lay a sufficiently precise groundwork for precise checkpointing.

Secondly, I present a PyTorch library that, given a sequential network, profiles the layers, solves for the optimal policy, and runs the network according to this policy.
Unfortunately, I have been unable to solve all of the implementation issues for running the network according to a policy, so thus far there is only a simulator.
I explain how these issues arise in PyTorch and demonstrate why the technique cannot be naturally implemented in Autograd.

I evaluate the efficacy of generalising the policy solver to precise per-layer costs by comparing the (simulated) peak memory and execution time
when and when not profiling.
I also investigate the extent to which bucketing memory affects the solver, required due to how large real-world budgets are.
Finally, I give a thorough qualitative evaluation of my work, appraise checkpointing with respect to other memory optimisations, and suggest directions for future work.

The research into checkpointing done by the Machine Learning community has rediscovered results from Automatic Differentiation.
Throughout this thesis, I refer to these resuts, especially when suggesting future work.